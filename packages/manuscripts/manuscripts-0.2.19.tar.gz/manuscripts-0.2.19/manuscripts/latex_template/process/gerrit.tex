For Gerrit data source, we use REI, that was defined in section \ref{overview}, to measure the efficiency in code review process. Chart below shows evolution of REI by quarters in order to visualize how changesets are being managed by the community.

\begin{tabular}{p{7cm} p{5cm}}
	\vspace{0pt} 
	\includegraphics[scale=.35]{figs/gerrit_bmi_reviews_general.eps}
	& 
	\vspace{0pt}
	\begin{tabular}{l|l}%
		\bfseries Period & \bfseries Closed/Subm. % specify table head
		\csvreader[head to column names]{data/gerrit_bmi_reviews_general.csv}{}% use head of csv as column names
		{\\\labels & \bmireviews}
	\end{tabular}
\end{tabular}

In terms of time and again for Gerrit, chart below shows the evolution of mean and median times--in days--to close a review (TTM, defined in section \ref{overview}).

\begin{tabular}{p{7cm} p{5cm}}
	\vspace{0pt} 
	\includegraphics[scale=.35]{figs/gerrit_days_to_merge_review_avg_gerrit_days_to_merge_review_median_general.eps}
	& 
	\vspace{0pt}
	\begin{tabular}{l|r|r|}%
		\bfseries Period & \bfseries Median & \bfseries Mean % specify table head
		\csvreader[head to column names]{data/gerrit_days_to_merge_review_avg_gerrit_days_to_merge_review_median_general.csv}{}% use head of csv as column names
		{\\\labels & \daystomergereviewmedian & \daystomergereviewavg}
	\end{tabular}
\end{tabular}