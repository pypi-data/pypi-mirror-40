To measure the efficiency of the process of software development in GitHub first we focus on BMI, that was defined in section \ref{overview}. Chart below shows evolution of BMI by quarters, allowing us to check how pull requests are being managed by the community in the last quarter compared to previous ones.

\begin{tabular}{p{7cm} p{5cm}}
	\vspace{0pt} 
	\includegraphics[scale=.35]{figs/github_prs_bmipr_general.eps}
	& 
	\vspace{0pt}
	\begin{tabular}{l|l}%
		\bfseries Period & \bfseries Closed/Subm. % specify table head
		\csvreader[head to column names]{data/github_prs_bmipr_general.csv}{}% use head of csv as column names
		{\\\labels & \bmipr}
	\end{tabular}
\end{tabular}

Besides, next chart deals with timing. It shows mean and median times to merge pull requests in GitHub (TTM, defined in section \ref{overview})--in days--for last quarter compared to previous ones.


\begin{tabular}{p{7cm} p{5cm}}
	\vspace{0pt} 
	\includegraphics[scale=.35]{figs/github_prs_days_to_close_pr_avg_github_prs_days_to_close_pr_median_general.eps}
	& 
	\vspace{0pt}
	\begin{tabular}{l|r|r|}%
		\bfseries Period & \bfseries Median & \bfseries Mean % specify table head
		\csvreader[head to column names]{data/github_prs_days_to_close_pr_avg_github_prs_days_to_close_pr_median_general.csv}{}% use head of csv as column names
		{\\\labels & \daystocloseprmedian & \daystoclosepravg}
	\end{tabular}
\end{tabular}
